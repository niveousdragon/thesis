\documentclass[journal]{elsarticle}
\usepackage{lineno,hyperref}
\usepackage{graphicx}
\usepackage{amsmath,amssymb,amsfonts}
\usepackage{algorithm}
\usepackage{algorithmic}
\usepackage{textcomp}
\usepackage{xcolor}
\usepackage{url}
% Russian language support
\usepackage[T2A]{fontenc}
\usepackage[utf8]{inputenc}
\usepackage[russian]{babel}
\modulolinenumbers[5]

\bibliographystyle{elsarticle-num}

\begin{document}

\begin{frontmatter}

\title{Снижение размерности фМРТ-данных методом лапласовых собственных карт для обнаружения вызванных стимулами изменений в активности мозга в состоянии покоя}


\begin{abstract}
Мозг в состоянии бодрствования активен даже в отсутствие целенаправленного поведения или значимых стимулов. Однако паттерны активности в состоянии покоя (RS) могут претерпевать долгосрочные изменения после воздействия предшествующих значимых стимулов. Данное исследование было направлено на разработку объективного метода обнаружения таких изменений в активности RS после воздействия эмоционально значимых стимулов. Для этой цели была использована функциональная магнитно-резонансная томография (фМРТ) активности мозга в состоянии покоя до и после приобретения и угашения экспериментально обусловленного страха. Группа здоровых добровольцев участвовала в трех сессиях фМРТ: RS до обусловливания страха, сессия угашения страха и RS непосредственно после угашения страха. Парадигма обусловливания страха состояла из трех нейтральных зрительных стимулов, сопряженных с частичным подкреплением слабым электрическим током. Были использованы как линейные, так и нелинейные методы снижения размерности для различения между начальным RS и RS после воздействия стимулов. Анализ главных компонент (PCA) как линейный метод снижения размерности показал значительно худшие результаты, чем нелинейные методы (Isomap, LLE, лапласовы собственные карты). С использованием метода обучения на основе многообразий лапласовых собственных карт удалось показать значимые различия между двумя RS на уровне отдельных участников. Это обнаружение было дополнительно улучшено путем сглаживания BOLD-сигнала с помощью вейвлет-анализа с множественным разрешением. Разработанный метод может улучшить различение функциональных состояний, собранных в лонгитюдных исследованиях фМРТ.
\end{abstract}

\begin{keyword}
фМРТ состояния покоя \sep BOLD-сигнал \sep Обусловленный страх \sep Снижение размерности \sep Лапласовы собственные карты \sep Анализ с множественным разрешением
\end{keyword}

\end{frontmatter}

\section{Введение}

Глобальная динамика мозга, генерируемая в покое, приводит к множеству наложенных паттернов активности, известных как сети состояния покоя (RSN). RSN воспроизводимы в больших выборках здоровых субъектов \cite{vandenheuvel2010,martynova2017} и значительно изменены при неврологических и психиатрических расстройствах \cite{hohenfeld2018,sheffield2016}. Оценка активности мозга в состоянии покоя (RS) на больших выборках показала более надежные результаты, чем фМРТ, индуцированная задачами, что подразумевает, что функциональные связи RS могут служить биомаркерами расстройств мозга \cite{satterthwaite2015,barch2017}. Таким образом, активность RS имеет решающее значение для понимания динамики и организации активности мозга в норме и при патологии.

Хотя RSN определяются на основе отсутствия целенаправленного поведения или значимых стимулов, важным вопросом является то, отражает ли их активность возможные остаточные изменения в покое после предшествующей активности, вызванной задачей. Этот вопрос особенно актуален для понимания механизмов тревожных и посттравматических стрессовых расстройств (ПТСР) \cite{graham2011}. Действительно, несколько исследований фМРТ показали повышенную функциональную связность (FC) некоторых областей в RS после обусловливания страха \cite{hahn2011,prater2013,baeken2014,russo2017,jung2018}. Данные RS фМРТ указывают на высокую корреляцию между тревожностью и изменениями в FC \cite{kim2011,belleau2018}, а также на повышенную FC у пациентов с ПТСР между областями мозга, которые являются ключевыми для обработки эмоциональных стимулов \cite{zhou2012}, даже спустя годы после воздействия стресса \cite{brown2014}.

Однако высокая размерность данных фМРТ осложняет их обычный анализ, основанный на статистическом параметрическом картировании \cite{friston1996} изменений уровня сигнала, зависящего от оксигенации крови (BOLD) \cite{poldrack2008}. Высокая размерность особенно проблематична в отсутствие начальной гипотезы, которая снижает размерность, фокусируясь только на интересующих областях.

Более того, в условиях высокой размерности данных ценные результаты могут быть потеряны из-за коррекции на множественные сравнения в отсутствие предварительно выбранных интересующих областей. Потеря потенциальных маркеров особенно критична при сравнении индивидуальных лонгитюдных данных или при сравнении небольших выборок данных при редких клинических заболеваниях. Кроме того, персонализированные подходы выходят на первый план в клинической нейронауке, поскольку становится важным выявлять различия в активности мозга индивидуума, а не статистические различия между группами, чтобы прояснить диагноз, эффективность лечения и реабилитации. С этой точки зрения методы снижения размерности данных могут улучшить эффективность различения и классификации изменений активности мозга как на индивидуальном, так и на групповом уровнях, особенно в случае подхода, управляемого данными.

Считается, что почти любые данные высокой размерности могут быть эффективно вложены в пространство значительно более низкой размерности — так называемое внутреннее многообразие. При поиске такого внутреннего пространства обычно предполагается, что нет априорных данных о положениях точек на внутреннем многообразии, то есть данные не имеют меток. Методы обучения на основе многообразий разработаны специально для анализа этой низкоразмерной структуры данных.

Методы снижения размерности наиболее широко распространены в анализе данных, который направлен на выявление внутренней структуры данных. В машинном обучении и компьютерном зрении они доказали свою достаточно высокую эффективность для классификации изображений и текста. Их применение к данным нейронауки также продемонстрировало их полезность. Например, методы обучения на основе многообразий (ML) показали, что они могут обнаруживать низкоразмерные динамические особенности в данных мультиэлектродных массивов \cite{cunningham2014,gao2015} и были полезны в анализе транскриптомики одиночных клеток для идентификации траекторий развития нейронных клеток \cite{tritschler2019,kharchenko2021}. В нейровизуализационных исследованиях методы ML чаще используются для задач классификации: различения между пациентами и здоровыми субъектами \cite{tolan2018,ma2019} или различных условий задач \cite{su2009,richiardi2013,suk2015,liu2017}.

Среди методов ML анализ главных компонент (PCA) является наиболее часто используемым методом для анализа данных фМРТ. PCA — это линейный метод снижения размерности, направленный на определение взаимно ортогональных направлений в пространстве данных, вдоль которых дисперсия максимальна. Таким образом, PCA позволяет выделить «существенные» независимые оси и спроецировать все данные в пространство меньшей размерности. Несколько исследований, использовавших PCA на данных фМРТ всего мозга, обнаружили, что различия FC могут описывать низкоразмерную динамику мозга \cite{zhong2009,cordes2002}. Однако PCA оптимален только для данных с гауссовым распределением; он игнорирует структуры со статистикой более высокого порядка, которые могут способствовать нелинейным преобразованиям \cite{hyvarinen2000,hyvarinen2004}. Таким образом, нелинейные методы снижения размерности могут быть более точными для различения паттернов FC в сложной динамике активности мозга.

Многие алгоритмы обучения на основе многообразий появились с середины 2000-х годов \cite{cunningham2014}. Они оказались полезными инструментами для анализа динамики нейронных популяций \cite{kemere2008}. Методы, основанные на данных одновременной записи активности многих нейронов, эволюционировали от исходной проблемы к исследовательскому анализу данных одной сессии записи. Их применение к данным нейровизуализации только начинается. Эти методы позволяют исследователям находить корреляции низкоразмерной динамики с различными поведениями, такими как переходы походки \cite{deng2016}, траектория движения головы \cite{rubin2014} и принятие решений \cite{machens2010,harvey2012,kobak2016}. Они также применимы для выявления специфической динамики активности мозга. Например, Isomap \cite{tenenbaum2000} может обнаруживать различные состояния при эпилепсии височной доли \cite{jackson2018}.

Учитывая данные, следует выбрать метод обучения на основе многообразий, который более эффективен для выявления его внутренней структуры. Методы снижения размерности, которые превосходят PCA в раскрытии базовой структуры FC в высокоразмерных сигналах мозга фМРТ, недостаточно изучены. Поэтому в данном исследовании было протестировано несколько хорошо известных нелинейных подходов ML, таких как Isomap, локально линейное вложение (LLE) и лапласовы собственные карты (LE), для обнаружения изменений в состоянии покоя, которые происходят после воздействия обусловливания страха. Также были исследованы различные методы предварительной обработки сигналов фМРТ для улучшения эффективности обнаружения.

\section{Методы}

\subsection{Участники и процедура}

Были проанализированы данные фМРТ состояния покоя 23 здоровых правшей-добровольцев (23,90 ± 3,93 года, 8 женщин), которые участвовали в исследовании лонгитюдных изменений функциональной связности мозга RS, таких как RS после обучения страху и угашения памяти о страхе (RS\_FE) по сравнению с исходными данными RS (RS\_0). Эти данные были получены в предыдущем исследовании \cite{tetereva2020}.

\subsection{Этическое заявление}

Протокол исследования соответствовал требованиям Хельсинкской декларации, и исследование было одобрено Этическим комитетом Института высшей нервной деятельности и нейрофизиологии Российской академии наук. Все участники предоставили письменное информированное согласие перед исследованием.

\subsection{Описание процедуры}

Процедура исследования была следующей: 1) начальное сканирование в состоянии покоя (RS\_0); 2) процедура обучения страху (FL) вне сканера (павловское обусловливание страха); 3) угашение страха (FE) и 4) сканирование в состоянии покоя (RS\_FE) после угашения страха. Время между сканированиями RS\_0 и RS\_FE составляло приблизительно 45 минут, а между FE и RS\_FE — 1--2 минуты. Во время сканирования RS участникам было предложено оставаться спокойными с закрытыми глазами и стараться не думать целенаправленно. Полное описание экспериментальной процедуры можно найти в нашей предыдущей работе \cite{tetereva2020}. В текущем исследовании было сосредоточено внимание на анализе RS1 и RS2, чередующихся с FL и FE, чтобы разделить два набора данных сканирования RS, полученных до и после воздействия эмоционально значимых стимулов.

\subsection{Обучение страху и угашение страха}

Процедура FL проводилась в отдельной комнате в поведенческой лаборатории. FL состояла из двух псевдослучайных последовательностей с коротким перерывом между ними. Для FL использовалась парадигма отсроченного обусловливания страха с частичным отрицательным подкреплением. Были представлены три визуальных (геометрических фигуры) условных стимула. Один тип стимулов всегда был нейтральным. За представлением каждого условного стимула следовал белый экран случайной продолжительности 8--12 с с джиттером 2 с. Два других стимула имели вероятности подкрепления 70\% и 30\% соответственно. Безусловным стимулом была слабая стимуляция электрическим током продолжительностью 500 мс, которая предъявлялась сразу после фигуры, когда появлялся белый экран. Сила стимуляции выбиралась индивидуально, чтобы быть терпимым, но болезненным стимулом. Перед каждым стимулом участники видели фиксационный крест продолжительностью 2 с. Продолжительность каждого условного стимула варьировалась случайным образом от 4 до 8 с с джиттером 2 с.

Вторая последовательность была такой же, как и первая, за исключением того, что вероятности подкрепления для условных стимулов были изменены на 30\% и 70\% соответственно. Общая продолжительность каждого блока FL составляла 8 минут 54 секунды.

Во время сессии FE те же стимулы предъявлялись, но в другом псевдослучайном порядке и с более продолжительной общей последовательностью (10 минут) и без US. Во время сессии FE добровольцев просили ожидать безусловный стимул, но с другим правилом подкрепления, чем в предыдущих двух сессиях.

\subsection{Сбор данных фМРТ}

Данные МРТ были собраны в Национальном исследовательском центре «Курчатовский институт» (Москва, Россия) с использованием 3T сканера (Magnetom Verio, Siemens, Германия), оснащенного 32-канальной головной катушкой. Функциональные изображения (300 объемов) были собраны с использованием $T_2^*$-взвешенной эхо-планарной визуализации (EPI) с фактором ускорения GRAPPA, равным 4, и следующими параметрами последовательности: TR 2000 мс, TE 20 мс, FA 90°; 42 среза, полученных в чередующемся порядке с толщиной среза 2 мм, промежутком между срезами 0,6 мм и полем зрения 200 мм с матрицей сбора данных 98 × 98. Кроме того, анатомические изображения были собраны с использованием последовательности T1-MPRAGE: TR 1470 мс, TE 1,76 мс, FA 9°; 176 срезов с толщиной среза 1 мм, промежутком между срезами 0,5 мм и полем зрения (FoV) 320 мм с размером матрицы 320 × 320.

\subsection{Предварительная обработка данных фМРТ}

Данные обеих сессий RS были обработаны с использованием MELODIC, части FSL (FMRIB's Software Library) \cite{jenkinson2012}. MELODIC — это программа многомерной исследовательской линейной оптимизированной декомпозиции на независимые компоненты (ICA). FSL использовалась для предварительной обработки данных фМРТ. Были выполнены следующие стандартные шаги: удаление первых пяти объемов каждого набора данных RS фМРТ для учета эффектов уравновешивания сигнала, коррекция движения с помощью MCFLIRT, которая применялась для коррекции движения головы субъекта путем выравнивания каждого объема функциональных данных со средним объемом с использованием аффинного преобразования с 6 степенями свободы (DOF), извлечение мозга из черепа с использованием BET, пространственное сглаживание с использованием гауссова ядра FWHM 4 мм для увеличения отношения сигнал/шум, нормализация интенсивности общего среднего всего 4D набора данных с помощью единого мультипликативного фактора и высокочастотная временная фильтрация (взвешенная по Гауссу аппроксимация прямой линии методом наименьших квадратов с сигмой = 60,0 с и регистрацией функциональных изображений на структурные изображения (T1) с использованием FLIRT с 12 DOF и обычным поиском. Структурное изображение T1 было зарегистрировано в изображении стандартного пространства (шаблон MNI-152 2 мм) с использованием аффинного преобразования с 12 DOF.

Искажения B0 были удалены с использованием полевых карт. Дальнейшая предварительная обработка включала MELODIC для извлечения отдельных независимых компонентов (38 компонентов). После ICA одного субъекта отдельные компоненты были классифицированы с использованием обученного классификатора FIX (FMRIB's ICA-based X-noiseifier). FIX был вручную обучен на 20 наборах данных из текущего исследования с использованием порога 20. На основе рекомендаций были протестированы пороги от 10 до 50 и выбран тот, который дает наименьшее количество ложноположительных результатов. Процедура автоматического удаления артефактов движения на основе ICA (ICA-AROMA) \cite{pruim2015} классифицировала компоненты данных и удаляла артефакты движения. Функциональные данные анализировались в пространстве субъекта.

Объемы с кадровым смещением (FD), превышающим 0,5 мм, и один объем до и два объема после были помечены как выбросы и заменены средним сигналом. Среднее FD после предварительной обработки составило 0,144 ± 0,073 мм. Данные участника отбрасывались, если 10\% объемов превышали 0,25 мм FD. Однако все предварительно обработанные данные участников были включены в анализ. На втором этапе были регрессированы средние сигналы из желудочков и белого вещества, глобальный сигнал и шесть параметров движения. Регрессия выполнялась с использованием частичной корреляции. Таким образом, вклад каждого мешающего фактора был удален из временного ряда с контролем всех других мешающих факторов. Предварительная обработка выполнялась с двойным подходом к удалению шума (как было описано выше), включая FIX и ICA-AROMA. Полученные в результате очищенные от шума компоненты MELODIC впоследствии использовались для получения предварительно обработанных функциональных данных.

\subsection{Парцелляция мозга для получения BOLD-сигнала}

Был использован атлас парцелляции Brainnetome, BNA \cite{fan2016}. Атлас включает 246 областей мозга на основе анатомической структуры и функциональной дифференциации областей. Была исключена область BN 94 из анализа, так как эта область отсутствовала у некоторых людей, у которых размер мозга был больше используемого поля зрения. Каждая маска была преобразована в индивидуальное пространство субъекта с использованием FLIRT FSL. Этот атлас использовался для извлечения усредненных BOLD-сигналов для каждой из 245 ROI.

\subsection{Аналитическая структура}

Основным подходом было снижение размерности данных от количества парцелл до эффективной. Значение BOLD-сигнала косвенно зависит от активности нейронов в интересующей структуре мозга (область интереса - ROI) \cite{poldrack2008}. Если рассматривать BOLD-ответ в одной ROI как отдельную переменную, состояние мозга в каждый момент может быть представлено как $D$-мерный вектор активности, где $D$ соответствует количеству исследуемых ROI и зависит от схемы парцелляции мозга. Вдоль каждой оси в таком пространстве хранится значение сигнала в определенной области. Таким образом, изменение BOLD-сигнала всего мозга может быть представлено как движение точки в этом многомерном пространстве.

В нашем исследовании 300 значений BOLD-сигнала из 245 ROI (парцелл BN) с шагом 2 с соответствовали 300 функциональным изображениям мозга, полученным с TR = 2 с, для каждой сессии сканирования. Другими словами, в каждой сессии было 300 точек в многомерном пространстве. Задача заключалась в том, чтобы как можно точнее разделить состояния мозга, связанные с различными сессиями (RS\_0 и RS\_FE). Для этой цели 300 точек из каждой сессии были объединены в единый набор данных из 600 точек, и метки классов были стерты. Затем точки были разделены на две группы с использованием снижения размерности.

\subsection{Снижение размерности}

Для сравнения различных подходов к обучению на основе многообразий были применены PCA и три нелинейных метода обучения на основе многообразий: Isomap, локально линейное вложение (LLE) и лапласовы собственные карты (LE). Мы отсылаем читателя к превосходным обзорам \cite{vandermaaten2009,sorzano2014,yan2018,liu2022} для более подробного описания разнообразия алгоритмов ML. В общем случае методы ML предполагают, что все данные лежат на одном многообразии уменьшенной размерности, встроенном в пространство большей размерности, то есть количество измерений данных.

\subsubsection{Isomap}

Isomap \cite{tenenbaum2000} — это нелинейное обобщение классического линейного метода, называемого многомерным масштабированием (MDS), от которого Isomap отличается нелинейным отображением. В этом методе попарное евклидово расстояние между векторами данных высокой размерности заменяется геодезическим расстоянием на многообразии данных (измеренным вдоль путей на нем). Расчет геодезических расстояний состоит из трех этапов:

1. Построение графа окрестностей, в котором каждая точка набора данных соединена с фиксированным количеством ее ближайших соседей.
2. Расчет кратчайших путей от каждой точки до всех остальных на построенном графе с использованием алгоритма Дейкстры. Это дает оценку геодезических расстояний.
3. Алгоритм MDS применяется к матрице попарных геодезических расстояний, полученной на втором этапе. Таким образом, мы получаем желаемое низкоразмерное представление данных.

\subsubsection{Локально линейное вложение}

В отличие от Isomap, LLE \cite{roweis2000} основан на локальных, а не глобальных свойствах данных. Этот метод предполагает, что многообразие локально линейно, то есть что каждая точка данных и ее соседи лежат на или близко к локально линейному участку многообразия. На первом шаге LLE назначает соседей каждой точке данных. Далее LLE реконструирует каждую точку данных как линейную комбинацию ее соседей. Чтобы найти низкоразмерное вложение данных, LLE ищет координаты, где точки могут быть хорошо реконструированы теми же линейными комбинациями их соседей. Таким образом, LLE сохраняет локальную геометрию данных высокой размерности в низкоразмерном вложении.

\subsubsection{Лапласовы собственные карты}

Лапласовы собственные карты \cite{belkin2003} — это нелинейный метод, основанный на спектральной теории графов. Подобно методу LLE, для каждой точки строятся $k$-ближайших соседей. Альтернативно можно соединить все точки данных, расположенные ближе, чем указанный порог расстояния. На втором этапе каждому ребру полученного графа присваивается вес; таким образом, мы получаем взвешенный граф. Существуют две стратегии взвешивания: двоичные веса (веса всех ребер равны 1) и веса в соответствии с функцией теплового ядра:

\begin{equation}
W_{ij} = \exp\left(-\frac{||x_i - x_j||^2}{t}\right)
\label{eq:heat_kernel}
\end{equation}

где $t$ — свободный параметр. Вложение затем получается путем вычисления собственных векторов лапласиана графа:

\begin{equation}
L = D - W
\label{eq:laplacian}
\end{equation}

где $D$ — диагональная матрица с $D_{ii} = \sum_j W_{ij}$. Собственные векторы, соответствующие наименьшим ненулевым собственным значениям, предоставляют координаты вложения.

Алгоритм лапласовых собственных карт может быть суммирован в 8 шагах:

\begin{enumerate}
\item Построить граф окрестностей, используя k-ближайших соседей или критерий расстояния $\epsilon$.
\item Выбрать веса: $W_{ij} = 1$ для простого графа или $W_{ij} = \exp(-||x_i - x_j||^2 / t)$ для функции теплового ядра.
\item Вычислить матрицу степеней $D$, где $D_{ii} = \sum_j W_{ij}$.
\item Вычислить лапласиан графа $L = D - W$.
\item Решить обобщенную задачу на собственные значения $L v = \lambda D v$.
\item Сортировать собственные значения в порядке возрастания: $0 = \lambda_0 \leq \lambda_1 \leq ... \leq \lambda_{n-1}$.
\item Использовать собственные векторы $v_1, ..., v_d$, соответствующие $d$ наименьшим ненулевым собственным значениям, в качестве координат вложения.
\item Вложить точку данных $x_i$ в d-мерное пространство с использованием координат $(v_1(i), v_2(i), ..., v_d(i))$.
\end{enumerate}

\subsection{Оценка внутренней размерности}

Для всех наборов данных была проведена оценка внутренней размерности с использованием различных методов. Для PCA была использована оценка из \cite{fukunaga1971}. Для Isomap была использована формула из \cite{vandermaaten2009}, основанная на остаточной дисперсии. Для LLE была использована формула из \cite{wang2011}, основанная на объясненной дисперсии. Наконец, для лапласовых собственных карт была использована формула из \cite{lee2010}, основанная на спектре собственных значений нормализованного лапласиана.

\subsection{Многомасштабное вейвлет-преобразование для сглаживания флуктуаций BOLD-сигнала}

Для улучшения отношения сигнал/шум (SNR) данных фМРТ был применен вейвлет-анализ с множественным разрешением \cite{mallat1999}. Дискретное вейвлет-преобразование (DWT) разлагает сигнал на аппроксимационные и детальные коэффициенты на нескольких масштабах. Использовалось семейство вейвлетов Добеши из-за его хороших свойств локализации как во временной, так и в частотной областях. Удаление шума выполнялось с использованием метода VisuShrink с мягким порогом. Стандартное отклонение шума на каждом масштабе оценивалось с использованием медианного абсолютного отклонения (MAD) детальных коэффициентов на этом масштабе.

В данном исследовании использовался вейвлет Добеши с 4 исчезающими моментами. Сигналы были разложены на 4 уровня на основе длины сигнала (300 точек). На каждом уровне были извлечены аппроксимационные и детальные коэффициенты. Коэффициенты ниже порога были удалены на основе уровней порога, определенных алгоритмом мягкого порога, а коэффициенты выше порога были сохранены. Наконец, реконструкция была выполнена обратным вейвлет-преобразованием с использованием модифицированных коэффициентов.

\subsection{Различение двух сессий состояния покоя}

Отношение среднего расстояния между точками разных классов к среднему расстоянию между точками одного класса (межклассовое-внутриклассовое расстояние, IID) использовалось в качестве меры разделения двух классов точек (соответствующих сессиям до и после воздействия). В данном исследовании было решено сосредоточиться на анализе IID, поскольку он измеряет расстояние между двумя популяциями «в общем», что представляет интерес.

Стандартные метрики качества бинарной классификации показывают, насколько хорошо алгоритм справляется с маркировкой тестовых данных, при условии, что метки классов для обучающих данных известны. Нас также интересовал другой вопрос — насколько хорошо данные, принадлежащие разным сессиям сканирования, были разделены в новом низкоразмерном пространстве. Вот почему IID был выбран в качестве основной метрики. Обратите внимание, что если мы присваиваем метки точкам данных случайным образом, IID не превысит 1,01 ± 0,01.

В дополнение к IID были рассчитаны точность и оценка ROC-AUC алгоритма кластеризации $k$-средних на данных вложения с $k = 10$. Результаты были кросс-валидированы с использованием десяти фолдов (10 неперекрывающихся разделений обучающих и тестовых данных, тестовый набор содержит 10\% данных в каждом разделении).

Все результаты были усреднены по субъектам после удаления выбросов, которые выделялись из среднего более чем на 2,5 стандартных отклонения.

\section{Результаты}

\subsection{Сравнение точности разделения данных фМРТ}

IID между двумя сессиями после применения PCA был лишь немного выше случайного: 1,014 ± 0,011 для простого PCA и 1,022 ± 0,015 для PCA с предварительным вейвлет-удалением шума. Однако вейвлет-удаление шума значительно увеличило точность и оценку AUC классификации, выполненной на вложениях PCA.

В отличие от линейных методов (PCA), нелинейные методы продемонстрировали значительно более высокие результаты. Среди них лапласовы собственные карты продемонстрировали самые высокие значения IID и точность классификации. В частности, лапласовы собственные карты с вейвлет-удалением шума достигли IID 2,2 ± 0,5, обеспечивая четкое разделение между двумя состояниями покоя. Добавление вейвлет-удаления шума последовательно улучшало производительность во всех методах.

В качестве дополнительного анализа была оценена внутренняя размерность (ID) многообразия данных фМРТ с использованием оценки максимального правдоподобия, предложенной \cite{levina2005}. Такая оценка дает $\text{ID} = 6.1 \pm 0.6$.

На рисунке 2 представлены оценки ошибок (точность классификации и площадь под кривой (ROC\_AUC)) как функции размерности вложения для различных методов снижения размерности. В отличие от линейных методов (PCA), нелинейные методы продемонстрировали значительно более высокие результаты как в 2, так и в 3 измерениях.

На рисунке 3 показаны значения RMSD, которые указывают, насколько точно методы обучения на основе многообразий производят свои вложения. Для всех методов более высокая размерность приводит к более точному вложению. Примечательно, что нелинейные методы имеют гораздо более низкий RMSD, чем PCA для малых размерностей (от 2 до 10), что означает, что они могут представлять больше информации о внутренней структуре данных в низкоразмерных вложениях.

\subsection{Вейвлет-удаление шума улучшает разделение данных BOLD-сигнала из разных сессий}

На рисунке 4 показано, как вейвлет-преобразование улучшает SNR. Панель (а) показывает пример исходного и отфильтрованного BOLD-сигнала временного ряда одной ROI. Панель (b) демонстрирует, как варьируются ошибки для различных конфигураций параметров вейвлета.

Улучшение разделения IID после добавления этапа предварительной обработки вейвлета составило $210\% \pm 50\%$ для случая одноэтапного снижения размерности и $290\% \pm 120\%$ для случая двухэтапного снижения (рис.~\ref{fmri:fig5},~\ref{fmri:fig7}). Интуитивно это означает, что в среднем расстояние между различными кластерами RS увеличилось в 2-3 раза по сравнению с базовым уровнем, когда было добавлено предварительное сглаживание сигнала с помощью DWT.

Было обнаружено, что одномерное пространство часто давало наилучшее разделение в соответствии с точностью классификации и оценкой IID. Чтобы исследовать этот вопрос, было проанализировано, что представляет низкоразмерная ось в случае одномерного вложения. Оптимальный уровень разложения был 4, что соответствует удалению флуктуаций быстрее приблизительно 0,1 Гц.

На рисунке 5 показаны примеры двумерных вложений, созданных различными методами снижения размерности. Разделение между состояниями RS\_0 (синий) и RS\_FE (красный) наиболее четко видно во вложениях лапласовых собственных карт, особенно с вейвлет-удалением шума.

На рисунке 6 представлены индивидуальные примеры, демонстрирующие изменчивость качества разделения среди участников. Было обнаружено, что для 19 из 23 участников удалось найти низкоразмерное представление, где эти два RS занимали бы разные области. Некоторые участники показывают четкую кластеризацию двух состояний («хорошие» примеры), в то время как другие показывают большее перекрытие («плохие» примеры).

На рисунке 7 показано, как вейвлет-удаление шума улучшает разделение между сессиями. Графы близости становятся более четко разделенными после применения вейвлет-преобразования.

На рисунке 8 показано систематическое исследование параметров вейвлета, демонстрирующее, что выбор порядка вейвлета и уровня разложения влияет на качество разделения.

Графы близости обеспечивают визуализацию структуры вложения. На рис.~\ref{fmri:fig1.1}--\ref{fmri:fig1.3} представлены низкоразмерные двумерные вложения двух RS для конкретного субъекта, полученные различными методами (PCA на рис.~\ref{fmri:fig1.1}, LLE на рис.~\ref{fmri:fig1.2} и LE на рис.~\ref{fmri:fig1.3}). Голубые точки соответствуют RS\_0, а красные точки соответствуют RS\_FE. В случае PCA (рис.~\ref{fmri:fig1.1}) RS образуют смешанное облако без четкого кластерного разделения. Таким образом, два состояния покоя неразличимы. Нелинейные алгоритмы, такие как LLE (рис.~\ref{fmri:fig1.2}) и LE (рис.~\ref{fmri:fig1.3}), обеспечивают лучшее разделение RS в низкоразмерном пространстве. LE показывают, что два кластера четко разделены в низкоразмерном пространстве для этого субъекта. На рис.~\ref{fmri:fig1.4}--\ref{fmri:fig1.6} дополнительно представлены матрицы смежности RS. Мы видим, что структура клеток в матрицах смежности из вложений LE является более блочной, чем в матрицах PCA.

Порядки вейвлета 3 и 4 и уровни разложения 3 и 4 дают наилучшие результаты разделения. Мы проанализировали влияние порядка вейвлета на разделение сессий и обнаружили, что высокоуровневый шум в BOLD-сигнале может вызывать дополнительные геометрические артефакты в графе близости и искусственно увеличивать внутриклассовое расстояние.

\subsection{Предварительное снижение размерности алгоритмом LE приводит к лучшему разделению состояний покоя}

Также был протестирован двухэтапный подход, при котором PCA сначала применялся для снижения размерности пространства с исходной 245 (количество областей мозга) до промежуточного $10$-мерного пространства, а затем использовалось обучение на основе многообразий для создания окончательных вложений. На рисунке 10 показано, что, хотя этот подход может улучшить вычислительную эффективность, он обычно приводит к более низкому качеству разделения по сравнению с прямым применением нелинейных методов. Такое снижение особенно помогло в случае алгоритма LLE. Для алгоритмов LE и Isomap улучшение не было столь заметным, но все же значимым.

На рисунке 11 сравниваются различные конструкции графов близости для алгоритма лапласовых собственных карт, показывая, что подход $k$-ближайших соседов с $k = 10$--$20$ дает оптимальные результаты.

Дополнительные рисунки S1 и S4 предоставляют дополнительные детали об алгоритмических реализациях и анализе чувствительности параметров.

\section{Обсуждение}

Данные 23 участников × 3 сессии сканирования были использованы для разделения RS\_0 от RS\_FE как для индивидуальных, так и для групповых данных. В результатах было показано, что неполные вложения могут быть найдены для разделения данных в низкоразмерном пространстве. Возможным объяснением этого результата является то, что низкоразмерные вложения сигналов позволили наблюдать тонкую систематическую структуру. Разделение индивидуальных данных кажется более впечатляющим, поскольку оно может обеспечить потенциал для будущего персонализированного подхода, нацеленного на определение индивидуального состояния мозга человека.

Линейный метод PCA не был успешным. Это подтверждает наше предположение о том, что взаимосвязь между нейронной активностью в различных областях мозга является нелинейной. Только три субъекта показали точность более 55\% на линейных компонентах, в то время как 20 из 23 субъектов показали точность более 55\% на нелинейных вложениях. Было получено низкоразмерное представление, где два состояния покоя были хорошо разделены, что делает возможным дальнейший дискриминантный анализ.

Было обнаружено, что алгоритм LE показывает лучшие результаты среди нелинейных методов обучения на основе многообразий. Подход Isomap был сопоставим с LE, в то время как алгоритм LLE был определенно хуже. Также стоит отметить большую разницу между нелинейными алгоритмами и классическим линейным PCA. Рис.~\ref{fmri:fig3}, представляющий RMSD (см. рис.~\ref{fmri:figS3}), то есть точность, с которой алгоритм восстанавливает структуру входных данных, особенно показателен в этом отношении. Очевидно, что нелинейные алгоритмы сохраняют структуру данных гораздо точнее, чем PCA.

В целом алгоритм LE работает аналогично методу диффузионных карт, который успешно использовался для анализа данных BOLD-сигнала \cite{margulies2016}. Применение этих алгоритмов менее затратно в вычислительном отношении, чем алгоритм Isomap. Кроме того, размерность низкоразмерного пространства, необходимая для достаточно точного представления исходных данных, лучше определяется для алгоритма LE, чем для Isomap.

Наилучшее разделение было достигнуто с использованием LE, когда данные проецировались в одномерное пространство. Такое поведение связано с особенностями алгоритма лапласовых собственных карт. Первый нетривиальный собственный вектор — это так называемый вектор Фидлера, который обеспечивает наилучшее линейное разделение на графе. Стоит отметить, что при $t = 0$ веса $w_{ij} = 1$ для всех связанных вершин. В этом случае алгоритм LE полностью эквивалентен методу диффузионных карт. Поэтому мы не можем интерпретировать значение скрытых переменных, соответствующих дальнейшим собственным векторам. Вложение в одномерное пространство можно рассматривать как проекцию активности мозга на определенное направление в исходном пространстве признаков. С нашей точки зрения, это направление максимально различает конфигурации активности мозга в состоянии покоя до и после воздействия значимых эмоциональных стимулов.

Также было обнаружено, что вейвлет-удаление шума дополнительно улучшает разделение состояний RS\_0 и RS\_FE. Было обнаружено, что для всех нелинейных алгоритмов предварительное сглаживание с использованием вейвлет-преобразования улучшало разделение состояний мозга. Наибольшее улучшение наблюдалось для «худшего» алгоритма, LLE, в то время как для оптимального алгоритма LE улучшение было относительно небольшим.

Результаты согласуются с предыдущими выводами об изменениях мозга, связанных с обусловливанием страха, которые включают изменения в лимбических и префронтальных областях, которые участвуют в обработке страха \cite{bahrami2019,dunsmoor2019,lange2020}. Кроме того, недавние исследования показали нелинейные динамические изменения активности мозга после задач обусловливания страха \cite{hong2020}. Также интересно, что были обнаружены изменения на уровне всего мозга, а не в отдельных областях, что означает, что уникальный стимул вызывает глобальное изменение во всей организации мозга.

Важно отметить, что наша задача имеет значительные ограничения. В целом изменения активности мозга после FE могут быть связаны не только с консолидацией памяти, но и с неспецифическим стрессом возбуждения или усталостью. Не регистрировались такие параметры, как гальванические кожные реакции или другие периферические показатели, которые могли бы быть поведенческими коррелятами изменений мозга. Не удалось соотнести изменения активности мозга после FE с поведенческими показателями. Чтобы доказать специфичность выявленных изменений, необходимо иметь контрольную группу с предъявлением изначально нейтральных стимулов. Это позволит отличить изменения, связанные с конкретным содержанием воспоминаний о страхе, от неспецифических эффектов, таких как возбуждение.

Также следует подчеркнуть, что было протестировано применение алгоритмов обучения на основе многообразий для обнаружения индивидуальных изменений мозга после значимого стимула. Это отличается от большинства применений машинного обучения, которые классифицируют активность мозга для разделения людей в соответствии с нозологией, поведенческими симптомами или личностными характеристиками. Обнаружение многообразий направлено на то, чтобы смотреть на данные «непредвзято», не зная классов, к которым принадлежат данные. Этот подход является исследовательским, помогая определить внутреннюю структуру данных, идентифицировать кластеры и выбросы, а также обнаружить неочевидные паттерны. Однако, если цель состоит в том, чтобы разработать классификатор, который может различать два или более класса, следует использовать другой подход.

Несмотря на успех нашего подхода, мы признаем, что другие методы ML могут быть не менее полезными. Были протестированы только три алгоритма обучения на основе многообразий с различными подходами к вычислению внутреннего многообразия. Другие алгоритмы ML могут давать разные результаты. Более того, комбинация нескольких подходов обучения на основе многообразий может давать лучшие результаты. Например, UMAP, разработанный для анализа транскриптома одиночных клеток \cite{mcinnes2018}, показал хорошие результаты при применении к данным нейровизуализации \cite{papadopoulos2021}. Эти подходы также должны быть протестированы для оценки функциональной динамики в фМРТ.

Используя методы обучения на основе многообразий, была продемонстрирована возможность визуализации и количественной оценки функционального состояния мозга в состоянии покоя. Наш подход может быть расширен и применен к функциональным сетям мозга. Интересный пример такого подхода был недавно описан в \cite{qiu2015}, где метод LLE использовался для изучения иерархически встроенных низкоразмерных структур, лежащих в основе динамических функциональных сетей.

\section{Выводы}

В нашем исследовании было продемонстрировано, что метод обучения на основе многообразий лапласовых собственных карт, применяемый к данным фМРТ, может выявлять различия между функциональной активностью мозга RS до и после воздействия субъекта на значимый стимул. Наши результаты также демонстрируют, что функциональные сети мозга RS сохраняют эффекты воздействия стимулов даже после временной задержки. Различия между RS четко наблюдались на уровне отдельных участников, что подразумевает потенциал методов обучения на основе многообразий для выявления новых маркеров динамики мозга. В дальнейших приложениях эта методология может быть расширена на персонализированный мониторинг фМРТ-биомаркеров психиатрических и неврологических расстройств.

\section*{Вклад авторов}

Те, кто задумал и разработал исследование, включают НП, АТ, ОМ и КА. АТ провела эксперименты. АТ и НП проанализировали данные. НП и ОМ написали статью.

\section*{Финансирование}

Работа выполнена при поддержке Российского научного фонда, грант № 17-78-30029 (Анализ связности, лонгитюдные изменения связности) и соглашение № 075-15-2020-801 (Обработка данных).

\section*{Благодарности}

Мы благодарны доктору С. Нечаеву и доктору А. Горскому за важные предложения и комментарии к рукописи.


\section*{Список литературы}

\bibliography{references}

\end{document}