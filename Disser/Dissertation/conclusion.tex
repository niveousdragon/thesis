\chapter*{Заключение}                       % Заголовок
\addcontentsline{toc}{chapter}{Заключение}  % Добавляем его в оглавление

%effdim
В данной работе представлен способ количественной оценки общей синхронизации активности большой популяции нейронов на основе эффективной размерности. Кроме того, впервые, насколько известно авторам, применена коррекция к спектрам корреляционных матриц, что особенно важно для экспериментальных записей с ограниченным временем. Применение разработанного алгоритма к данным эксперимента, в котором мыши исследовали новую среду, продемонстрировало, что эффективная размерность значительно снижается во время периодов движения. Это связывается с синхронной активностью нейронов гиппокампа, модулируемой движением. Предполагается, что данный анализ может стать ценным количественным инструментом для изучения синхронизации больших нейронных групп.


%intdim
В данной работе исследовано поведение внутренней размерности нейронной активности и проведено сравнение с мерами размерности, основанными на линейности. Подтверждено, что активность гиппокампальных популяций лежит на сильно искривлённом низкоразмерном многообразии, размерность которого остаётся почти постоянной во времени, в отличие от линейных мер. При исследовании масштабирования $ВР(N)$ с размером системы обнаружен степенной закон с малым критическим показателем $\alpha$, который не объясняется структурой временного ряда данных. Это масштабирование является характеристикой коллективной нейронной активности, и его количественная оценка может обеспечить дальнейшее понимание режимов нейронной активности на популяционном уровне.

%% Согласно ГОСТ Р 7.0.11-2011:
%% 5.3.3 В заключении диссертации излагают итоги выполненного исследования, рекомендации, перспективы дальнейшей разработки темы.
%% 9.2.3 В заключении автореферата диссертации излагают итоги данного исследования, рекомендации и перспективы дальнейшей разработки темы.
%% Поэтому имеет смысл сделать эту часть общей и загрузить из одного файла в автореферат и в диссертацию:

Основные результаты работы заключаются в следующем.
\input{common/concl}
И какая-нибудь заключающая фраза.

Последний параграф может включать благодарности.  В заключение автор
выражает благодарность и большую признательность научному руководителю
Иванову~И.\,И. за поддержку, помощь, обсуждение результатов и~научное
руководство. Также автор благодарит Сидорова~А.\,А. и~Петрова~Б.\,Б.
за помощь в~работе с~образцами, Рабиновича~В.\,В. за предоставленные
образцы и~обсуждение результатов, Занудятину~Г.\,Г. и авторов шаблона
*Russian-Phd-LaTeX-Dissertation-Template* за~помощь в оформлении
диссертации. Автор также благодарит T., без которого эта работа не была бы закончена.
